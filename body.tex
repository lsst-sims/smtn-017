\section{Metrics}



\subsection{Wide Area Metrics}

\noindent{\bf Parallax}: A measure of the parallax precision in the best 18k square degrees of the survey (probably, the WFD area).  \\
{\bf Proper Motion}: A measure of the proper motion precision in the best 18k square degrees of the survey (probably, the WFD area). \\
{\bf Fast Microlensing events}: Fraction of microlensing events between 5 and 10 days in duration that are recovered. \\
{\bf Slow Microlensing events}: Fraction of microlensing events between 60 and 90 days in duration that are recovered.\\
{\bf SRD fO value}: The area that reaches 825 visits (all filters). Should be around 18,000 square degrees. \\
{\bf Bright NEOs}: Fraction of H<=16 Near Earth Asteroid objects discovered. \\
{\bf Faint NEOs}: Fraction of NEO objects with H<=22 recovered.  \\
{\bf TNOs}: Fraction of H<=6.0 Trans Neptunian Objects discovered.  \\
{\bf SNe}: The number of type Ia supernovae that are observed up to the redshift completeness limit. \\
{\bf SNe, zlim}: The redshift completeness limit for Ia SNe. \\
{\bf 3x2}: Figure of merit for the 3x2 correlation. Only uses i-band. \\
{\bf Weak Lensing}: Number of visits in i-band that are suitable for weak lensing observations, includes an extinction cut. \\
{\bf Transients, KNe}: The PrestoKNe metric. This metric generates 10,000 events. Unfortunately, in the baseline only around 400 of the events are detected, so we expect a run-to-run shot noise of ~5\%. The metric returns two scores ("P" and "S") which are highly correlated. Here we use the total of the score S.  \\
{\bf XRB}: X-ray binary events. This metric generates 10,000 events, and here we check how many pass the early detection criteria. \\
{\bf Brown Dwarf}:  This metric computes the volume we can detect L7 Brown Dwarfs in. Because brown drawfs are very red objects, this metric will be mostly sensitive to depth in the $izy$\ filters.\\

The raw metric values are output at the end of this jupyter notebook: \url{https://github.com/lsst-sims/smtn-017/blob/main/plots/v20.ipynb}

\subsection{DDF metrics}

xxx--coadded depth, SNe. Maybe some AGN. Need photo-z


\section{Baseline Evolution}


\begin{figure}
\plottwo{plots/baselines/retro_baseline_v2_0_10yrs_Count_observationStartMJD_HEAL_SkyMap.pdf}{plots/baselines/baseline_retrofoot_v2_0_10yrs_Count_observationStartMJD_HEAL_SkyMap.pdf}
\plottwo{plots/baselines/baseline_v2_0_10yrs_Count_observationStartMJD_HEAL_SkyMap.pdf}{plots/baselines/baseline_v2_1_10yrs_Count_observationStartMJD_HEAL_SkyMap.pdf}
\caption{Evolution of the baseline footprint, showing total number of visits in all filters.\label{fig:baseline_foot}}
\end{figure}

\begin{figure}
\plottwo{plots/baselines/retro_baseline_v2_0_10yrs_Year3_5Count_night_gt_1278_375000_and_night_lt_1643_625000_HEAL_SkyMap.pdf}{plots/baselines/baseline_retrofoot_v2_0_10yrs_Year3_5Count_night_gt_1278_375000_and_night_lt_1643_625000_HEAL_SkyMap.pdf}
\plottwo{plots/baselines/baseline_v2_0_10yrs_Year3_5Count_night_gt_1278_375000_and_night_lt_1643_625000_HEAL_SkyMap.pdf}{plots/baselines/baseline_v2_1_10yrs_Year3_5Count_night_gt_1278_375000_and_night_lt_1643_625000_HEAL_SkyMap.pdf}
\caption{Evolution of the baseline rolling strategy. \label{fig:baseline_roll}}
\end{figure}

\begin{figure}
\plottwo{plots/baseline2_radar.pdf}{plots/baseline2_radar_zoom.pdf}
\caption{The science impact of our latest baseline simulation. Plots are the same at different zoom levels. Metric values have been normalized to the v2.1 baseline.  The major change of increasing the depth of the bulge leads to a huge increase in microlensing and XRB metrics. \label{fig:baseline2_radar}}
\end{figure}

\begin{figure}
\epsscale{0.6}
\plotone{plots/baselines_mags.pdf}
\epsscale{1}
\caption{Median coadded depths for the baseline-like simulations, normalized to the v2.1 baseline. The v2.1 baseline is $\sim$0.1 mags shallower than most previous baselines, mainly due to relaxing the weight put on taking observations at low airmass. \label{fig:baseline_mags}}
\end{figure}

The baseline footprints are shown in Figure~\ref{fig:baseline_foot}, while a check on how the different baselines roll is shown in Figure~\ref{fig:baseline_roll}. Science metrics and coadded depths are in Figures~\ref{fig:baseline2_radar} and~\ref{fig:baseline_mags}. 

Shifting the footprint so the bulge is covered with a higher number of visits radically increases the microlensing and XRB metrics. The new footprint also improves the 3x2 metrics by increasing the amount of low extinction area covered in the WFD. The addition of rolling cadence boosts the SNe metric. The solar system metrics see a modest improvement moving to the v2.1 baseline. 

The proper motion, parallax, and brown dwarf volume all see a slight decrease over previous baselines.

\section{v2.0 Results}



\subsection{Bluer}

\begin{table}
\caption{Relative number of visits per filter}
\begin{tabular}{lrrrrrr}
\toprule
{} &   u &   g &    r &    i &   z &   y \\
baseline\_v2.0    & 3.2 & 4.0 & 10.0 & 10.0 & 9.0 & 9.0 \\
bluer\_indx0\_v2.0 & 3.3 & 5.7 & 10.0 & 10.0 & 9.0 & 9.5 \\
bluer\_indx1\_v2.0 & 3.8 & 5.2 & 10.0 & 10.0 & 9.0 & 9.5 \\
\end{tabular}
\label{table:blue}
\end{table}

We run a few simulations where we increase the fraction of time spent in $u$\ and $g$\ filters. The relative number of observations in each filter is listed in Table~\ref{table:blue}. Figure~\ref{fig:bluer_radar} shows no major improvement for any science case when increasing the blue fraction. We had expected SNe could benefit from added blue observations, however, since we still heavily favor red filters in bright time, the cadence of blue observations does not change enough to help the SNe metric. Similarly, TDEs often want more $u$\ observations, but the number added here is not sufficient to help the TDE metrics.


\begin{figure}
\epsscale{0.5}
\plotone{plots/bluer.pdf}
\plotone{plots/bluer_mags.pdf}
\epsscale{1}
\caption{The science and coadded depth impact of shifting to bluer filters. We see no significant gains in any particular science case.\label{fig:bluer_radar}}
\end{figure}

\subsection{Long $u$}

\begin{figure}
\epsscale{0.5}
\plotone{plots/long_u.pdf}
\plotone{plots/long_u_mags.pdf}
\epsscale{1}
\caption{The science and depth impact of taking longer $u$\ observations. We see no significant gains in any particular science case.\label{fig:long_u}}
\end{figure}


The long\_u1 run increases the $u$-band exposure time to 50s leaving the relative number of observations the same as the baseline while long\_u2 increases the exposure time but decreases the total number of observations in $u$. 

Looks like we don't really have a metric that is sensitive to increasing $u$-depth. We do see significant codded depth increase in $u$\ in both cases. If there is no case for the higher cadence, the long\_u2 run seems to have extra depth "for free".


\subsection{Rolling}


\begin{figure}
\epsscale{0.6}
\plotone{plots/rolling_ns.pdf}
\epsscale{1}
\caption{Various rolling cadence experiments with different rolling strengths and number of rolling bands. Nothing jumps out as obviously superior to the baseline rolling strategy. \label{fig:rolling}}
\end{figure}


\begin{figure}
\epsscale{0.6}
\plotone{plots/rolling_more.pdf}
\epsscale{1}
\caption{Comparing a run with no rolling and one with an additional season of rolling. If we do not roll at all, the number of SNe has a dramatic drop, as to short transients. \label{fig:rolling_more}}
\end{figure}


\begin{figure}
\epsscale{0.6}
\plotone{plots/rolling_bulge.pdf}
\epsscale{1}
\caption{Rolling in the bulge area. No significant gains, and it can hurt microlensing. \label{fig:rolling_bulge}}
\end{figure}

We test rolling with different fractions of the sky (half, third, sixth), and two different strengths (50\% and 90\%). 
Our baseline of half sky at 80-90\% seems fairly close to ideal. Rolling with smaller area can boost SNe and TDE metrics, but at significant penalty for proper motions and faint NEOs.

Turning off rolling gives a modest boost to astrometry metrics, but is very bad for almost all transients. 

The `roll\_early` tests the impact of adding an additional rolling season. This gives a nice boost to SNe and transients, with a small impact on proper motions. I guess I can point out that if the survey extends slightly, that proper motion precision can be recovered, but once the transients are gone, they are gone forever.

Adding rolling in the bulge doesn't have any perks. We might need more bulge-specific transient metrics.


Figure~\ref{fig:rolling_more} shows having no rolling has a significant impact on SNe and KNe while starting rolling early to gain an additional season of rolling gives a boost with only a slight penalty to proper motions. 

We have not tested rolling in the dusty plane (or SCP or NES). There may be benefits to having some higher cadence observations in the plane, but we currently do not have metrics that would show that.

The LMC and SMC are currently observed as part of the WFD region, and thus get higher and lower cadence seasons as part of rolling. 


\begin{figure}
\epsscale{0.5}
\plotone{plots/rolling/noroll_v2_0_10yrs_Year2_5Count_night_gt_913_125000_and_night_lt_1278_375000_HEAL_SkyMap.pdf}
\plotone{plots/rolling/roll_early_v2_0_10yrs_Year2_5Count_night_gt_913_125000_and_night_lt_1278_375000_HEAL_SkyMap.pdf}
\plotone{plots/rolling/rolling_all_sky_ns2_rw0_9_v2_0_10yrs_Year2_5Count_night_gt_913_125000_and_night_lt_1278_375000_HEAL_SkyMap.pdf}
\epsscale{1}
\caption{No rolling, early rolling, and rolling including the bulge.\label{fig:rolling_normal}}
\end{figure}


\begin{figure}
\epsscale{0.5}
\plotone{plots/rolling/rolling_bulge_6_v2_0_10yrs_Year2_5Count_night_gt_913_125000_and_night_lt_1278_375000_HEAL_SkyMap.pdf}
\plotone{plots/rolling/rolling_bulge_ns2_rw0_5_v2_0_10yrs_Year2_5Count_night_gt_913_125000_and_night_lt_1278_375000_HEAL_SkyMap.pdf}
\plotone{plots/rolling/rolling_bulge_ns2_rw0_8_v2_0_10yrs_Year2_5Count_night_gt_913_125000_and_night_lt_1278_375000_HEAL_SkyMap.pdf}
\plotone{plots/rolling/rolling_bulge_ns2_rw0_9_v2_0_10yrs_Year2_5Count_night_gt_913_125000_and_night_lt_1278_375000_HEAL_SkyMap.pdf}
\epsscale{1}
\caption{Rolling in the bulge, different number of bands and different rolling strengths.}
\end{figure}

\begin{figure}
\epsscale{0.5}
\plotone{plots/rolling/six_rolling_ns6_rw0_5_v2_0_10yrs_Year2_5Count_night_gt_913_125000_and_night_lt_1278_375000_HEAL_SkyMap.pdf}
\plotone{plots/rolling/six_rolling_ns6_rw0_9_v2_0_10yrs_Year2_5Count_night_gt_913_125000_and_night_lt_1278_375000_HEAL_SkyMap.pdf}
\epsscale{1}
\caption{6-band rolling. This is so extreme the entire WFD area is not guaranteed to be covered every year.}
\end{figure}

\begin{figure}
\epsscale{0.5}
\plotone{plots/rolling/rolling_ns2_rw0_5_v2_0_10yrs_Year2_5Count_night_gt_913_125000_and_night_lt_1278_375000_HEAL_SkyMap.pdf}
\plotone{plots/rolling/rolling_ns2_rw0_9_v2_0_10yrs_Year2_5Count_night_gt_913_125000_and_night_lt_1278_375000_HEAL_SkyMap.pdf}
\plotone{plots/rolling/rolling_ns3_rw0_5_v2_0_10yrs_Year2_5Count_night_gt_913_125000_and_night_lt_1278_375000_HEAL_SkyMap.pdf}
\plotone{plots/rolling/rolling_ns3_rw0_9_v2_0_10yrs_Year2_5Count_night_gt_913_125000_and_night_lt_1278_375000_HEAL_SkyMap.pdf}
\epsscale{1}
\caption{Half- and one-third-sky rolling.}
\end{figure}



\subsection{Presto and long gaps}

The presto runs look to gather 3 observations in a night on various time scales. The long\_gaps simulations try to ensure time scales of several hours are observed.

The variations we try. 1) Varying the goal length of the gap between 1.5 and 4 hours, 2) taking in near pairs (g+r, r+i, i+z) or mixed pairs (g+i, r+z, i+y) 3) Try to gather triplets half the time (every-other night).

\begin{figure}
\plottwo{plots/presto_gap.pdf}{plots/presto_gaps_mags.pdf}
\plottwo{plots/presto_gaps_mix.pdf}{plots/presto_mix_gaps_mags.pdf}
\plottwo{plots/presto_half_gaps.pdf}{plots/presto_half_mags.pdf}
\plottwo{plots/presto_half_mix.pdf}{plots/presto_half_mixed_mags.pdf}

\caption{ Science and depth results of different simulations observing triplets in a night. Top two rows show observing triplets every night, bottom two rows show observing triplets every other night. \label{fig:presto}}
\end{figure}


Results for observing triplets in a night are shown in Figure~\ref{fig:presto}. As expected, the KNe transient metric benefits from observing triples. Also as expected, only triples with gaps of 3-4 hours show improvements.  

Observing triples every night results in much shallower u band final coadded depth (most likely because the triples are executing in dark time, forcing u observations into gray and bright). Triples and long gaps force observations to be taken at higher airmass in general than the baseline, so all of the simulations in these groups have shallower final coadded depths than the baseline.

All the simulations gathering triples greatly reduced the number of SNe recovered. Even if triples are only attempted on half the nights, SNe, faint NEOs, and astrometry all see significant reductions in their metrics.


xxx--need to double check what the astrometry SRD values are, the most aggressive presto runs might be bumping up against it.

\begin{figure}
\plottwo{plots/presto_sne.pdf}{plots/presto_astrom.pdf}
\caption{The SNe and astrometry metrics for the various presto runs. Baseline circled in red. \label{fig:presto_metics} }
\end{figure}


\begin{figure}
\plottwo{plots/long_gaps.pdf}{plots/long_gaps_mags.pdf}
\caption{Observing long gaps for the entire survey.}
\end{figure}

\begin{figure}
\plottwo{plots/long_gaps_delayed.pdf}{plots/long_gaps_delayed_mags.pdf}
\caption{Observing long gaps for the second half of the survey.}
\end{figure}


\subsection{Vary Galactic Plane}

xxx--Varying the amount of time spent in the plane. This may be obsolete now with the v2.1 galactic plane footprint simulations. 


\subsection{Vary North Ecliptic Spur}

\begin{figure}
\plottwo{plots/vary_nes1.pdf}{plots/vary_nes2.pdf}
\caption{Results from varying the amount of time spent in the NES. Left plot shows NES fractions lower than the baseline, right panel shows NES fractions larger than the baseline. The metrics show relatively little variation, but we expect more noticeable changes in other solar system metrics. \label{fig:vary_nes}}
\end{figure}

\begin{figure}
\plottwo{plots/vary_nes_mags.pdf}{plots/vary_nes_mags2.pdf}
\caption{Results from varying the amount of time spent in the NES. Left plot shows NES fractions lower than the baseline, right panel shows NES fractions larger than the baseline.\label{fig:vary_nes_mags}}
\end{figure}


Many of the solar system metrics are fairly insensitive to the fraction of time spent on the NES. I'll leave it to the solar system collaboration to make the case for where they think the optimal fraction is. Might be some potential to shave a little bit of time off of NES. The science gains from going from the baseline of 30\% down to something like 15\% are minimal though.


\subsection{Microsurveys}

\begin{figure}
\epsscale{0.6}
\plotone{plots/north_stripe.pdf}
\plotone{plots/north_stripe_mags.pdf}
\plotone{plots/footprints/north_stripe_v2_0_10yrs_Count_observationStartMJD_HEAL_SkyMap.pdf}
\epsscale{1}
\caption{Impact of adding a northern stripe to the footprint. \label{fig:north_stripe} }
\end{figure}



\begin{figure}
\plotone{plots/carina.pdf}
\caption{Impact of including the Carina microsurvey.  \label{fig:carina}}
\end{figure}


\begin{figure}
\plotone{plots/local_gals.pdf}
\epsscale{0.4}
\plotone{plots/footprints/local_gal_bindx0_v2_0_10yrs_Count_observationStartMJD_HEAL_SkyMap.pdf}
\plotone{plots/footprints/local_gal_bindx1_v2_0_10yrs_Count_observationStartMJD_HEAL_SkyMap.pdf}
\plotone{plots/footprints/local_gal_bindx2_v2_0_10yrs_Count_observationStartMJD_HEAL_SkyMap.pdf}
\epsscale{1}
\caption{Impact of including the local galaxies microsurvey. \label{fig:local_gals}}
\end{figure}

\begin{figure}
\plotone{plots/roman.pdf}
\caption{Impact of observing the Roman field.  \label{fig:roman}}
\end{figure}


\begin{figure}
\plotone{plots/short_exp.pdf}
\caption{Impact of observing short exposure times.  \label{fig:short_exp}}
\end{figure}

\begin{figure}
\plotone{plots/smc_movie.pdf}
\caption{Impact of observing the SMC in movie mode.  \label{fig:smc}}
\end{figure}


\begin{figure}
\plotone{plots/twilight.pdf}
\plotone{plots/twi_neo21.pdf}
\caption{Impact of observing NEOs in twilight time. Surprisingly reduces the fraction of faint NEOs recovered. Left panel shows the v2.0 run with 1s twilight exposures and the right shows v2.1 with 15s twilight exposures. \label{fig:twilight}}
\end{figure}

\begin{figure}
% from https://github.com/lsst-sims/sims_featureScheduler_runs2.0/blob/main/microsurveys/twilight_neo/check_twi.ipynb
\epsscale{0.6}
\plotone{plots/twilight_neo_nightpattern1v2_0_Nvisits_as_function_of_Alt_Az_note!twilight_neo_HEAL_SkyMap.pdf}
\plotone{plots/twilight_neo_nightpattern1v2_0_Nvisits_as_function_of_Alt_Az_notetwilight_neo_HEAL_SkyMap.pdf}
\plotone{plots/twilight_neo_nightpattern1v2_0_Nvisits_as_function_of_Alt_Az_notetwilight_neo_and_night10_HEAL_SkyMap.pdf}
\epsscale{1}
\caption{Number of visits in Alt,Az projection for the twilight NEO microsurvey. The top panel shows the standard 30s visit observations, middle panel shows 1s observations taken in twilight, and the bottom panel shows the twilight observations for one night. \label{fig:twi_maps}}
\end{figure}


ToO: Observing 10 or 50 ToOs per year. Minor impact on the science metrics as most ToO observations are in the WFD area.


Carina:  Observing the Carina star forming region intensely for a week per year, only 2,354 visits total.  Very minor impact, as expected.


local\_gals:  Observing local galaxies to extra depth in $gri$ to various levels. Note that these requested extra visits were assuming outdated baseline depths from minion\_1016, so may not be feasible anymore.

multi\_short:  Taking multiple short exposures. Note this may no longer be feasible with the latest constraints on shutter motions and heat generation.

north\_stripe:  Adding coverage to the north. Very minor impact on most metrics, but it would probably only help us recover a handful of ToO events. And we probably want to keep the ToO chasing in the WFD area as much as possible anyway.

roman: Observing RGES field for microlensing. Indeed, makes the fast microlensing metric go up. Virtually zero impact on the rest of the survey. Looks like these were 0.07\% of the total number of visits. 

short\_exp: Covering the sky with short exposures. Note this may no longer be feasible with the latest constraints on shutter motions and heat generation.

smc\_movie: Continuous observations of the SMC. This was only 2 nights on 15s exposures in g for 2780 visits. 

twilight\_neo:  Using short (1s) exposures in twilight to look for NEOs by pointing at high airmass in the direction of the sun. Note this may no longer be feasible with the latest constraints on how often the shutter can move without generating too much heat. There is little to no gain in the fraction of NEOs recovered with these simulations. We have also run a twilight neo survey with 15s exposures in v2.1 and find similar results. The v2.1 simulations recover 0.2-1\% of H=20 Vatiras objects. The 1s sims recover 0.5-3.5\% of H=20 Vatiras objects.--xxx maybe make a plot showing Vatiras performance.

virgo\_cluster:  Adding the Virgo cluster to the WFD footprint. This was such a minor change that we adopted it into the v2.1 baseline.


\section{v2.1 Runs}

\subsection{Baseline}

\begin{figure}
\plottwo{plots/map_labels.pdf}{plots/time_fractions/baseline_v2_1_10yrs_Count_night_note_not_like_DD_HEAL_SkyMap.pdf}
\caption{The left shows different regions used for the v2.1 baseline. Right shows the resultin number of visits. A breakdown is listed in Table~\ref{table:regions}.}
\end{figure}

\begin{table}
\caption{Number and fraction of visits for the v2.1 baseline. \label{table:regions}}
\begin{tabular}{lrr}
      Label &  N visits &  Percent \\
      \hline
   (None)    &   16687.1 &     0.80 \\
    LMC\_SMC &   42291.7 &     2.03 \\
      bulge &   85322.1 &     4.09 \\
dusty\_plane &  131940.0 &     6.33 \\
    lowdust & 1598178.2 &    76.77 \\
        nes &   89348.5 &     4.29 \\
        scp &   17371.9 &     0.83 \\
      virgo &    6647.0 &     0.31 \\
       DDFs &   93962.0 &     4.51 \\
\end{tabular}
\end{table}


\subsection{Galactic Plane Runs}

\begin{figure}
\plottwo{plots/footprints/pencil_fs1_v2_1_10yrs_Count_observationStartMJD_HEAL_SkyMap.pdf}{plots/footprints/pencil_fs2_v2_1_10yrs_Count_observationStartMJD_HEAL_SkyMap.pdf}
\plottwo{plots/footprints/plane_priority_priority0_9_pbf_v2_1_10yrs_Count_observationStartMJD_HEAL_SkyMap.pdf}{plots/footprints/plane_priority_priority0_6_pbf_v2_1_10yrs_Count_observationStartMJD_HEAL_SkyMap.pdf}
\plottwo{plots/footprints/plane_priority_priority0_4_pbf_v2_1_10yrs_Count_observationStartMJD_HEAL_SkyMap.pdf}{plots/footprints/plane_priority_priority0_1_pbf_v2_1_10yrs_Count_observationStartMJD_HEAL_SkyMap.pdf}
\caption{The various galactic plane footprints simulated.}
\end{figure}


\begin{figure}
\epsscale{0.6}
\plotone{plots/plane_pbf.pdf}
\plotone{plots/plane_pbt.pdf}
\plotone{plots/plane_pencil.pdf}
\epsscale{1}
\caption{Looking at simulations with different galactic plane coverage. Top panel shows different footprints, middle panel shows the same footprints plus pencil beam surveys, and the bottom shows only pencil beam surveys. \label{fig:gal_plane_radar}}
\end{figure}


We ran a variety of footprint variations requested by the community. These runs have some issues because they can make it difficult for the scheduler to identify large contiguous regions of sky to observe in blocks. Figure~\ref{fig:gal_plane_radar} shows that while we can see some improvement in the microlensing metrics, we currently lack metrics that show improvements in expanding Galactic plane coverage.



\subsection{Good Seeing}


\begin{figure}
\plotone{plots/good_seeing.pdf}
\caption{Observing some filters in good seeing conditions every year. }
\end{figure}


These runs attempt the observe the entire WFD area in "good seeing" conditions each year. This has already been incorporated into the baseline for the $gri$ filters (3 observations per year), so the relative changes here are minimal. We attempted to add $u$ to the list of filters getting good seeing observations, however there does not seem to be enough dark time to make this feasible.


% Table from https://github.com/lsst-sims/sims_featureScheduler_runs2.1/blob/main/good_seeing/seeing_check.ipynb
\begin{table}
\caption{Area (sq deg) with good seeing in first year}
\begin{tabular}{lrrrrrr}
                      run &    u &     g &     r &     i &     z &     y \\
                      \hline
            baseline\_v2.1 & 3996 & 16814 & 23462 & 24304 & 21610 & 21577 \\
  good\_seeing\_gsw0.0\_v2.1 & 8459 & 13990 & 21741 & 22890 & 22433 & 22453 \\
 good\_seeing\_gsw20.0\_v2.1 & 2353 & 17730 & 24491 & 24231 & 21928 & 21125 \\
  good\_seeing\_gsw3.0\_v2.1 & 4090 & 16641 & 23338 & 24319 & 22591 & 21487 \\
good\_seeing\_u\_gsw3.0\_v2.1 & 8828 & 16390 & 23130 & 23479 & 22060 & 21330 \\
\end{tabular}
\end{table}

\subsection{Different Standard Exposure Time}


\begin{figure}
\plottwo{plots/shave_shorter.pdf}{plots/shave_longer.pdf}
\caption{Trying different standard exposure times.  }
\end{figure}


We try standard exposure times from 20-40s.  The SNe metric seems surprisingly peaked at 30-32s for standard exposure times. This might be due to how they quantize redshift limits, since the number of SNe returned by the metric are the sum below a redshift completeness limit, a shift of 0.025 in the redshift limit could make a large change in N SNe($<z_{lim}$), but the actual total number of SNe would change more smoothly. 

Note, the boost in the WL metric is artificial since it is based on number of observations and thus does not take into account the lower SNR of shorter exposures. 


I'm not sure why proper motion and parallax benefit from shorter exposure times. Could be because the metric assumes no degeneracy between fit parameters. 

\subsection{Suppress Repeat Observations}


\begin{figure}
\plotone{plots/no_repeat.pdf}
\caption{Suppressing repeat visits within a night. A boost for SNe, and an unexpected boost for the KNe metric. }
\end{figure}


This series of runs looks at preventing more than two observations to a point within a night. As expected, this helps SNe. Unexpectedly, this also boosts the KNe transient metric. We need to check why this happens and if it is real since the KNe metric should have a 5\% scatter.


\subsection{DDF Season Length}

% From https://github.com/lsst-sims/sims_featureScheduler_runs2.1/blob/main/ddf_accourd/illustrate.ipynb
\begin{figure}
\plotone{plots/ddf_nvis_t.pdf}
\caption{Examples of different DDF observing strategies. The season length, low cadence fraction, and low cadence rate are all varied. \label{fig:ddf_acourd}}
\end{figure}


\begin{figure}
\plotone{plots/ddf_season_length.pdf}
\caption{Shifting from the baseline to pre-scheduled DDF observations.}
\end{figure}

\subsection{Twilight NEO}

\begin{figure}
\plotone{plots/twi_15s.pdf}
\caption{Science impact of dedicating twilight time to 15s visits looking near the sun.  \label{fig:twi_15}}
\end{figure}

Similar to the v2.0 twilight NEO survey, here we dedicate some variable fraction of twilight time to search for NEO objects. For these simulations, we use 15s visits. 

\section{Metrics That Would Help}

xxx--currently looking at running the photo-z metric to get more info on filter distribution in WFD and the DDFs.

xxx--We have some AGN metrics, I'm not sure they show much at the moment. We may need to configure, modify them a bit to get more info. These would be good for both the WFD and DDF areas. 


\section{Executive Summary}

\begin{itemize}
   \item{The v2.1 baseline is a very minor change over v2.0, adding the Virgo cluster to the WFD footprint which slightly lowers the median coadded depth.}
   \item{We find no major improvement in taking more observations in the $ug$\ filters, but should confirm the final filter depth distributions with a robust photo-z metric.}
   \item{Increasing the total time spent in $u$\ has detrimental effects to some science metrics. We currently don't see any science improvements from shifting to 50s $u$ exposures, but the final $u$ coadded depth increases $\sim$0.15 mags and there is a small increase in the SRD fO metric.}
   \item{Starting rolling early looks to be a significant improvement for some metrics with a very minor penalty for astrometry. Note that if the survey ran an additional 11th year, the proper motion penalty should be eliminated.}
   \item{The SSSC should look in depth to confirm the impact of changing the NES fraction. Seems like the current level or very slightly less time is probably optimal.}
   \item{The baseline currently comes close to covering the entire WFD area in good seeing conditions each year in $gri$. Adding $u$\ to good seeing does not seem feasible.}
   \item{Taking steps to suppress (extra) repeat visits within a night seems like it could help multiple science cases. It is tough to tell if the KNe metric increase is "real", or if the v2.1 baseline is simply a $\sim 2\sigma$ low realization.}
   \item{There are proposed footprints for more Galactic Plane coverage. Covering the galactic plane more can boost microlensing metrics, but additional science metrics for the Galactic plane would be helpful.}
   \item{We have tried a variety of twilight NEO surveys. They can increase the fraction of Vatrias objects recovered, but further simulations are planned for v2.2.}
   \item{{\bf The biggest tension appears to be between long and short timescale transients (where short is of order a day). While all the simulations identify many transients, the ability to classify short timescale transients hurts the light curve coverage of long transients. }}


\end{itemize}

